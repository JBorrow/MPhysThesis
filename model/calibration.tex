To ensure that the equation of state is continuous, it must be calibrated by considering when the star formation is quenched and the gas begins to behave as an isothermal gas again.
One way this can be realised is by considering the Jeans' length at the transition (for a full derivation of the below please see Appendix \ref{app:F}).

At the point where the thermal pressure and the trublent pressures are equal, their jeans lengths must also be equal, such that
$$
     L_{J, T} = L_{J, \sigma}~,
$$
which leads to the expression for the equlibrium temperature of the gas
\begin{equation}
    T_{eq} = \sigma^2_{eq} \left(\frac{3 \pi}{10 k_b}\right)~,
    \label{eqn:teq}
\end{equation}
where $k_b$ is the Boltzmann constant.

Using the dispersion relation from \citet{martizzi_supernova_2015}, the relation between equilibrium and critical surface density is
$$
  T_{eq} = (1.8)^2 \left(\frac{f}{F}\right)^{6/5} G^{4/5} P_f^{2/5} f_g^{-4/5} \Sigma_{g, eq}^{2/5} \left(\frac{3\pi}{10k_b}\right)~.
$$

This surface density can also be thought of as being given by the pressure when star formation is quenched by the UV background.
The surface density at which star formation is quenched is found to be $\approx 10 \msun \pc^{-2}$ from various sources, including the data from \citet{bigiel_star_2008} and the theoretical work of \citet{schaye_star_2004}.
Using the known temperature of the ISM in the warm phase of $10^{4}$ K, the supernovae efficiency,
\begin{equation}
    F = \left(\frac{(1.8)^2 \mu}{k_b T}\right)^{5/6} f P_f^{1/3} G^{2/3} f_g^{-2/3} \Sigma_{g, \mathrm{crit}}^{1/3}~.
    \label{eqn:Fcalib}
\end{equation}
Note that this now means that $F$ scales with the gas fraction (with $F \propto f_g^{-2/3}$).
For Milky-Way type values ($f_g = 0.1$) and the critical surface density of $10 \msun \pc^{-2}$, $F=0.5$.
