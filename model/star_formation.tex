\begin{figure}
    \centering
    \includegraphics[trim=0.5cm 0 1cm 1cm, width=\textwidth]{bigiel_fig.pdf}
    \caption{The star formation law from Equation \ref{eqn:martsfr} plotted on top of the data from \citet{bigiel_star_2008}. The dashed lines are contours of constant $\fgas$, i.e. more mass is added to the disk to increase $\sg$, whereas the solid lines have constant $\Sigma = \sst + \sg$, i.e. the disk trades out stellar mass for gas mass to increase the star formation rate. The constant $\fgas$ lines stop when the region reaches $\fgas=1$. Note that all lines on this plot take $F=1$ for simplicity and consistency; the true lines may vary from the ones plotted here (see \S \ref{sec:calibF} for more information), however it is clear that this model reproduces some trends in the data. In particular, the M51 data is extremely well fit by a $\Sigma \approx 100 \msun \pc^{-2}$ constant surface density fit. Note the cutoff in star formation at around $\sg \approx 10 \msun \pc^{-2}$. \citet{schaye_star_2004} argues that this is due to the UV background radiation in the galaxy preventing the gas from coalescing to form stars; this will be used in \S \ref{sec:calibF} to calibrate the supernovae efficiency $F$.}
    \label{fig:bigielwithmart}
\end{figure}

For a disk in hydrostatic eqilibrium, the supporting pressure can be written (see Appendix \ref{app:pT} and \citet{ostriker_maximally_2011} for a full derivation),
\begin{equation}
\label{eqn:hydeq}
p_H \approx \pi G \sg (\Sigma)~,
\end{equation}
where $\Sigma$ is the total surface density $\Sigma = \sg + \sst$.

The pressure generated from supernoave in the disk can be written (from dimensional analysis, see Appendix \ref{app:pH} and \citet{martizzi_supernova_2015})
\begin{equation}
\label{eqn:turb}
p_T = F \sfr \pfm,
\end{equation}
where $F$ is a factor of order unity that accounts for cancellation of momentum from interactions within the disk and the overall efficiency of the winds that the supernova drives, $P_{fin}$ is the average momentum injected per supernovae, and $m_*$ is the average mass of gas consumed per supernova candidate (here $m_* = 100 \msun$).

These two expressions are equal when the disk is in equlibrium giving a star formation law
\begin{equation}
\label{eqn:martsfr}
\sfr = \frac{\pi G}{F \pfm} \sg \Sigma~.
\end{equation}
This law is plotted against the data from \citet{bigiel_star_2008} in Figure \ref{fig:bigielwithmart} for various values of total surface density and gas fraction $f_{gas} = \sg/\Sigma$.
Similar results for the star formation law are found by \citet{ostriker_maximally_2011, faucher-giguere_feedback-regulated_2013, martizzi_supernova_2016}, and many more throughout the literature.
The $F$ paramter can be calibrated by considering the star formation cutoff in the $\sfr$, $\sg$ plane that is seen in \citet{bigiel_star_2008}, and the theoreitcal work of \citet{schaye_star_2004} (see \S \ref{sec:calibF} for more details).
The $P_{fin}/m_*$ parameter is calculated by using small-scale simulations.
