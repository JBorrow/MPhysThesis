For a disk in hydrostatic eqilibrium, the supporting pressure can be written (see appendix and \citet{ostriker2011} for a full derivation),
\begin{equation}
\label{eqn:hydeq}
p_H \approx \pi G \sg (\Sigma)~,
\end{equation}
where $\Sigma$ is the total surface density $\Sigma = \sg + \sst$.

The pressure generated from supernoave in the disk can be written (from dimensional analysis, see appendix and \citet{martizzi2015})
\begin{equation}
\label{eqn:turb}
p_T = F \sfr \pfm,
\end{equation}
where $F$ is a factor of order unity that accounts for cancellation of momentum from colliding supernovae shells, $P_{fin}$ is the average momentum injected per supernovae, and $m_*$ is the average mass of gas consumed per supernova candidate (here $m_* = 100 \msun$).

These two expressions are equal in equlibrium and as such a star formation law
\begin{equation}
\label{eqn:martsfr}
\sfr = \frac{\pi G}{F \pfin} \sg \Sigma~,
\end{equation}
which is plotted against the data from \citet{bigiel2008} in Figure \ref{fig:bigielwithmart} for various values of total surface density and gas fraction $f_{gas} = \sg/\Sigma$.
Similar results have been reported by \citep{ostriker2011}.
