The \citet{toomre_gravitational_1964} parameter, $Q_{gas}$ for the gas in the galaxy, is a measure of the stability of the disk.
For a thin disk with a negligeable vertical profile, a value of $Q<1$ implies that the disk is unstable with the shear forces due to differential rotation overcoming the internal pressure forces of the gas.
For a value of $Q>1$, the disk is overstable, with the pressure forces being more than enough to offset the differential rotation.
Whether real galaxy disks should be stable or not during their evolution is still a topic of debate in the community, however \citet{hopkins_stellar_2012} finds that real galaxy disks are \emph{marginally stable}, with a value of $Q \approx 1$.

The \citet{toomre_gravitational_1964} parameter is given as
\begin{equation}
    Q_{gas} = \frac{c_s \kappa}{\pi G \Sigma_Q}~,
    \label{eqn:Q}
\end{equation}
with $c_s$ the sound speed, $\kappa$ the epicyclic rotation frequency (taken to be $\sqrt{2} v/R$ in accordance with \citet{livermore_resolved_2015} with $v$ the particle velocity and $R$ its galactocentric radius) and $\Sigma_Q$ a combination of the gaseous and stellar surface densities of the disk.
It is important to note that the stellar and gaseous components of the disk contribute differently to the disks stability \citep{rafikov_local_2001}, and so here
\begin{equation}
    \Sigma_Q = \sg  + \left(\frac{2}{1 + f_\sigma^2}\right)\sst~,
    \label{eqn:sgq}
\end{equation}
with $f_\sigma = \sigma_*/\sigma_g \approx 2$ for real galaxies \citep{korchagin_local_2003}.

It is possible to predict the form of $\Sigma(r)$ expected as a function of galactocentric radius in the above model, by using the underlying dark matter profile to give $\kappa$, and the equation of state to give $c_s$ for a given $Q_{gas}$.
This is shown for the NFW profile used in the N-body simulations in Figure \ref{fig:toomreqthr}.

\subsubsection{Stable masses}

Given a dark matter halo profile, it is possible to find a stable gas mass of the disk by integrating over the surface density.
For the NFW profile, integration bounds must be chosen due to the profile being non-normalizable (here are [0, 30] kpc).
For a disk of $\fgas=0.1$, $F \approx 0.5$, and a dark matter halo with $M = 1.5\times10^{12} \msun$, the two models return the following for the mass contained within 30 kpc of the galactic centre:
\begin{itemize}
    \item This model: $1 \times 10^9 \msun$
    \item Isothermal monatomic gas at $10^4$ K: $2 \times 10^{10}\msun$
\end{itemize}
These numbers are largely within the bounds of a typical disk galaxy, for example the Milky Way has a gaseous disk with a mass of $\approx 5\times10^9 \msun$ \citep{licquia_improved_2013}.

\begin{figure}[t]
    \centering
    \includegraphics{toomre_q_theory.pdf}
    \caption{Predicted $\sg(r)$ for two models; on the left the equation of state found above and on the right the default equation of state in Gadget-2 (assuming a gas temperature of $10^4$ K). The grey line shows the predicted curve for $Q=1$. Assuming that the disks self-stabilise at $Q\approx1$, the surface density of stable galaxy in which the disk is supported by turbulence from supernovae should have a considerably lower surface density than one in which the gas behaves adiabatically. For more details on the underlying dark matter profile, see \S \ref{sec:sims}}
    \label{fig:toomreqthr}
\end{figure}

