This surface star formation law can be transformed into an effective equation of state by using the observation of \citet{schaye2004} that $\sg \approx \rho_g L_J$ (see Section \ref{sec:kslaw}).
In this calse, thermal pressure is neglected, and so the turbulent Jeans' length,
\begin{equation}
\label{eqn:jeans}
L_J = \rho_g^{1/2} \sigma \sqrt{\frac{9}{8G}}~,
\end{equation}
where $\sigma$ is the velocity dispersion.
is used (see Appendix for derivation).
\citet{martissi2015} finds that the velocity dispersion injected by supernovae is
\begin{equation}
\label{eqn:martdisp}
\sigma = 1.8 \left(\frac{f}{F}\right)^{{3/5}} G^{{2/5}} P_{fin}^{{1/5}} \fgas^{{-2/5}} \sg^{{1/5}}~.
\end{equation}
Using the epxression for the surface density above, and that $p = \rho_g \sigma^2$, the equation of state
\begin{equation}
\labeql{eqn:eos}
p = 4.5\left(\frac{f}{F}\right)^{{3/2}} G^{{3/4}} P_{fin}^{{1/2}} \fgas^{-1} \rho_\mathrm{g}^{{5/4}}~,
\end{equation}
is given (see Appendix for a more detailed derivation).
Here, $f$ is a factor that takes into account momentum cancellation from supernovae shells colliding.

This equation of state is plotted, for several values of $f/F$, along with various other equations of state in Figure \ref{fig:eos}.

