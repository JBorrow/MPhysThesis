\begin{figure}
    \savebox{\graphicsbox}{\includegraphics[]{eos_fig.pdf}}
    \leavevmode\rlap{\usebox{\graphicsbox}}
    \begin{minipage}[b]{0.42\wd\graphicsbox}
        {\caption{The equation of state (Equation \ref{eqn:eos}) is shown in solid green (for the values in \S \ref{sec:calibF}), with the EAGLE equation of state shown in dot-dash yellow. The equation of state that is shown in dotted blue was found to provide stable galaxies without extra physics by \citep{springel_cosmological_2003}. In grey dashed lines the various isothermal cutoffs (i.e. when star formation is quenched) for the various models are shown. The variation if cutoff density allows the supernova-driven equation of state to reproduce the one used in the EAGLE simulations well by giving a slightly higher normalization.}\label{fig:eos}}
\end{minipage}\hspace*{1\wd\graphicsbox}
\end{figure}

This surface star formation law can be transformed into an effective equation of state by using the observation of \citet{schaye_model-independent_2001} that $\sg \approx \rho_g L_J$.
In this case, thermal pressure is neglected, and so the turbulent Jeans' length,
\begin{equation}
\label{eqn:jeans}
L_J = \rho_g^{-1/2} \sigma \sqrt{\frac{9}{8G}}~,
\end{equation}
where $\sigma$ dominates over the sound speed and is the local turbulent velocity dispersion and $\rho_g$ is the local 3D gas density.
\citet{martizzi_supernova_2015} finds that the velocity dispersion injected by supernovae is
\begin{equation}
\label{eqn:martdisp}
\sigma = 1.8 \left(\frac{f}{F}\right)^{{3/5}} G^{{2/5}} P_{fin}^{{1/5}} \fgas^{{-2/5}} \sg^{{1/5}}~.
\end{equation}
Using the expression for the surface density above, and that $p = \rho_g \sigma^2$, the equation of state
\begin{equation}
\label{eqn:eos}
p = 4.5\left(\frac{f}{F}\right)^{{3/2}} G^{{3/4}} P_{fin}^{{1/2}} \fgas^{-1} \rho_\mathrm{g}^{{5/4}}~,
\end{equation}
for the gas in this model is found (see Appendix \ref{app:eos} for a more detailed derivation).
Here, $f$ is a factor that takes into account momentum cancellation from supernovae shells colliding.
The equation of state is plotted against various alternatives in Figure \ref{fig:eos}. A few remarks on this equation of state:
\begin{itemize}
    \item The inverse dependence of pressure on $\fgas$ correctly predicts that high-$z$ galaxies will have a lower level of support, leading to a high degree of clumping \citep{bournaud_long_2014}.
    \item This model predicts a low pressure support compared to other equations of state that are designed to produce stable galaxies, such as \citet{springel_cosmological_2003}. It does, however, reproduce the EAGLE equation of state well.
    \item Whilst there may initially appear to be a large number of extraneous factors, all of these ($f$, $F$, $P_{fin}$) are all fixed through either small scale simulations or observations, so there is a relatively small range of possible normalizations.
\end{itemize}
