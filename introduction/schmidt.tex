The Kennicutt-Schmidt star formation law, relating the surface star formation rate ($\dot{\sst}$) to the surface density of gas ($\sg$), is widely used throughout the literature.
The index of the Kennicutt-Schmidt law, $n$, is found empirically by many authors \citep{kennicutt_star_1989, kennicutt_jr._star_2007, bigiel_star_2008} to be in the region of $n=1.4$, such that
\begin{equation}
\label{eqn:ks}
\sst \propto \sg^n.
\end{equation}

The Kennicutt-Schmidt law originates from the original Schmidt star formation law  \cite{schmidt_rate_1959} that focuses on 3D densities such that
\begin{equation}
\label{eqn:s}
\dot{\rho}_* \propto \rho_g^{n_s}~,
\end{equation}
where $\dot{\rho}_*$ is the star formation rate and $\rho_g$ is the gas density.
It is considerably easier to measure surface densities within real galaxies, however, and as such the Kennicutt-Schmidt law is much more widely used.
In simulations, though, a 3D law is far more convenient, and as such it would be profitable to have a way of relating the two. 
This connection comes from the work of \citet{schaye_model-independent_2001}, which shows that in equilibrium the jeans column density $\Sigma_{g, J} \equiv \rho_g L_J$ (with $L_J$ the Jeans' length) is approximately the surface gas density in a (stable) disk,
\begin{equation}
\label{eqn:s04}
\sg \approx \Sigma_{g, J} \equiv \rho_g L_J~.
\end{equation}
