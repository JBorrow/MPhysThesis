Throughout the literature, the Kennicut-Schmidt star formation law, relating the surface star formation rate ($\sst$) to the surface density of gas ($\sg$), is widely used.
The index of the Kennicut-Schmidt law, $n$, is found empirically by many authors to be in the region of $n=1.4$, such that
\begin{equation}
\label{eqn:ks}
\sst \propto \sg^n.
\end{equation}

The Kennicut-Schmidt law originates from the original Schmidt star formation law that focuses on 3D densities such that
\begin{equation}
\label{eqn:s}
\dot{\rho}_* \propto \rho_g^{n_s}~,
\end{equation}
where $\dot{\rho}_*$ is the star formation rate and $\rho_g$ is the gas density.
It has proved difficult, however, to measure these 3D quantities within real galaxies and as such the Kennicut-Schmidt law is much more widely used.
In simulations, thought, a 3D law is far more convenient, and as such it would be profitable to have a way of relating the two. 
This connection comes from the work of \citet{schaye2004}, which shows that in equilibrium the jeans column density $\Sigma_{g, J} \equiv \rho_g L_J$ (with $L_J$ the Jeans' length) is approximately the surface gas density in a disk,
\begin{equation}
\label{eqn:s04}
\sg \approx \Sigma_{g, J} \equiv \rho_g L_J~.
\end{equation}
