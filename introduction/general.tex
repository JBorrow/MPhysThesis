The interstellar medium, as a major component, has a huge influence on the evolution and dynamics of galaxies.
Whilst the interstellar medium has a long history of being studied observationally, theoretically, and computationally, linking these together has been a challenge.
The work of \citet{schaye_star_2004} made a significant contribution in allowing surface gas densities, which are easily observed, to be connected to 3D densities, which are easily computed, in disk galaxies.

In this work, observations from \citet{bigiel_star_2008} are used to calibrate a surface star formation law, which leads to an effective equation of state when combined with the small-scale simulations of \citet{martizzi_supernova_2015}.
The central thesis here is that the turbulent pressure generated from supernovae balances the weight of the galactic disk when it is in equilibrium, a model which has been proposed numerous times over the years \citep{silk_feedback_1997, ostriker_maximally_2011, faucher-giguere_feedback-regulated_2013, martizzi_supernova_2016}.
This work then diverges from the aforementioned literature by using the effective equation of state as a subgrid specification in whole galaxy disks.

The report is laid out as follows: in what remains of Chapter \ref{sec:introduction}, the background to the work is discussed, along with some concepts that are used throughout the report and a short discussion on the structure of the ISM.
In Chapter \ref{sec:model}, the theoretical underpinning of the supernovae-driven model is given,along with the calibrations that are provided by \citet{martizzi_supernova_2015}, and predicted stable masses and surface density profiles are developed.
In Chapter \ref{sec:simulations}, an overview of the N-body simulations is given.
In Chapter \ref{sec:analysis}, a deeper analysis of the simulations is discussed.
In Chapter \ref{sec:discussion}, conclusions about the model are drawn, and in Chapter \ref{sec:appendix}, a number of useful appendices are given.
