The interstellar medium (ISM), as a major component, has a huge influence on the evolution and dynamics of galaxies.
Whilst the ISM has a long history of being studied observationally, theoretically, and computationally, linking these together has been a challenge.
The work of \citet{schaye_model-independent_2001} made a significant contribution in allowing surface gas densities, which are easily observed, to be connected to 3D densities, which are easily computed, in disk galaxies.

This key link between theory and observation is used to create a subgrid model, implemented through an equation of state, that uses the turbulent pressure from supernovae to support galaxy disks.
The idea that turbulence from supernovae provides pressure support to galaxy disks is not a new one; \citep{silk_feedback_1997, ostriker_maximally_2011, faucher-giguere_feedback-regulated_2013, martizzi_supernova_2016} however it has not been used, along with calibration simulations, to create a feedback-regulated subgrid model for use in N-body simulations.
To do this, high-resolution, small-scale simulations are required to calibrate the interactions of multiple supernovae and the dispersion that they provide \citep{martizzi_supernova_2015}.
This model is physically motivated and can be applied to hydrodynamical simulations, with it being tested here with a modified version of the Gadget-2 simulation code \citep{springel_cosmological_2003, springel_cosmological_2005}.

The report is laid out as follows: in what remains of Chapter \ref{sec:introduction}, background theory and motivation is presented.
In Chapter \ref{sec:model}, the theoretical underpinning of the supernovae-driven model is given, along with the inclusion of calibrations that are provided by \citet{martizzi_supernova_2015}, and predicted stable masses and surface density profiles are developed.
In Chapter \ref{sec:simulations}, an overview of the N-body simulations and their resolution dependence is shown, with a stability analysis of the galaxy disks being given in Chapter \ref{sec:analysis}.
In Chapter \ref{sec:discussion}, conclusions about the model are drawn and avenues for further work are given in Chapter \ref{sec:future}.
