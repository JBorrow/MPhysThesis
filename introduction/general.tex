The interestellar medium, as a major component, has a huge influence on the evolution and dynamics of galaxies.
Whilst the interstellar medium has a long history of being studied observationally, theoretically, and computationally, linking these together has been a challenge.
The work of \citet{schaye2004} made a significant contribution in allowing surface gas densities, which are easily observed, to be connected to 3D densities, which are easily computed, in disk galaxies.

In this work, observations from \citet{bigiel2008} are used to calibrate a surface star formation law, which leads to an effecive equation of state when combined with the small-scale simulations of \citet{martizzi2015}.
The central thesis here is that the turbulent pressure generated from supernovae balances the weight of the galactic disk when it is in equilibrium, a model which has been proposed numerous times over the years \citep{silk1997, ostriker2011, faucher-giguere2013, martizzi2016}.
This work then diverges from the aforementioned literature by using the effective equation of state as a subgrid specification in whole galaxy disks.
