The formation and evolution of galaxies is a central topic in modern astrophysics, and is often studied through the use of numerical simulations.
Increasingly, these simulations utilise a full hydrodynamical code \citep{vogelsberger_introducing_2014,schaye_eagle_2015} rather than the more traditional semi-analytical approach where baryonic properties are fit using a model to a N-body dark matter only simulation \citep{baugh_primer_2006, bower_breaking_2006}.
As such, the understanding of the detailed properties of gas within galaxies and their application to simulations is now a key area of research.

Stellar feedback is one of the keys to our understanding of gas dynamics on large scales.
Typically, the way that star formation and supernovae feedback is implemented in simulations is to provide a `kick' to a small area when a certain amount of stellar mass is formed to model a supernova (henceforth `stochastic supernovae').
The `kick' leads to the injection of $10^{51}$ ergs of energy (given for every $55 \msun$ formed) spread over several particles (or regions, in the case of an AMR code) in the simulation which generally have masses of $10^5 \msun$ or more \citep{tasker_simulating_2006, joung_dependence_2009, hummels_adaptive_2012, hopkins_meaning_2013, becerra_interstellar_2014}.
However, an alternative treatment is possible, whereby small-scale simulations are used to calibrate macroscopic parameters used in the underlying subgrid model of the gas, implemented by modifying the equation of state.
