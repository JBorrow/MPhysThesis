The formation and evolution of galaxies is a central topic in modern astrophysics, and is often studied through the use of numerical simulations.
Increasingly, these simulations utilise a full hydrodynamical code \citep{schaye_eagle_2015,vogelsberger_introducing_2014} rather than the more traditional semi-analytical approach whereby galaxy populations are fit to a dark matter only simulation \citep{baugh_primer_2006, bower_breaking_2006}.
As such, the understanding of the detailed properties of gas within galaxies and their application to simulations is now a key area of research
In this work, a physically motivated model of the gas dynamics and star formation in a disk galaxy that can be applied to hydrodynamical simulation is presented, along with tests using a modified version of the GADGET-2 simulation code \citep{springel_cosmological_2003, springel_cosmological_2005}.

Typically, the way that star formation and supernovae feedback is implemented in simulations is to provide a `kick' to a small area when a certain amount of stellar mass is formed to model a supernoa.
This leads to the $10^{51}$ ergs of energy (given for every $55 \msun$ formed) being modelled by the large particles (or regions, in the case of an AMR code) in the simulation which generally have masses of $10^5 \msun$ or more \citep{tasker_simulating_2006, joung_dependence_2009, hummels_adaptive_2012, hopkins_meaning_2013, becerra_interstellar_2014}.
In this work, an alternative treatment is proposed, whereby small-scale simulaitons are used to calibrate macroscopic parmaeters used in the underlying subgrid model of the gas, implemented by modifying the equation of state.
