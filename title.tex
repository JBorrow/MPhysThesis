\pagenumbering{gobble}

\begin{titlepage}

\thispagestyle{empty}

\author{Josh Borrow\\
\vspace{10mm}
\emph{Supervisor: Prof. Richard Bower}\\
\vspace{10mm}}
\date{\today}
\title{Towards a Physical Model of the Interstellar Medium\\
\vspace{4mm}
\large \emph{Can Supernovae Stabalise a Galaxy?}
\vspace{20mm}}
\maketitle

\vspace{20mm}

\begin{center}
\emph{{\small Submitted in partial satisfaction of the requirements for the degree}}\\
\emph{{\small FF3N ``Physics \& Astronomy (4 Years)" at Durham University.}}
\end{center}

\vspace{40mm}
\begin{abstract}
The interstellar medium (ISM) of disk galaxies is a complex and critical subject, however physically motivated models have proved elusive due to the hierarchical structure of the ISM. A simple model for the interstellar medium is presented that balances the weight of a galactic disk with the internal pressures created from star formation and the eventual supernovae of high mass stars. This is achieved by extracting macroscopic parameters from calibration simulations with $\approx$ 1 pc resolution. For such a simple model, it shows a high level of convergence with previous results. The model is tested in simulations of isolated disk galaxies using a modified version of the GADGET-2 SPH code.
\end{abstract}

\end{titlepage}

\newpage

\pagenumbering{arabic}
