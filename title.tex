\pagenumbering{gobble}

\begin{titlepage}

\thispagestyle{empty}

\author{Josh Borrow\\
\vspace{10mm}
\emph{Supervisor: Prof. Richard Bower}\\
\vspace{5mm}}
\date{\today}
\title{Towards a Physical Model of the Interstellar Medium\\
\vspace{4mm}
\large \emph{Can Supernovae Stabilise a Galaxy?}
\vspace{10mm}}
\maketitle

\vspace{10mm}

\begin{center}
\emph{{\small Submitted in partial satisfaction of the requirements for the degree}}\\
\emph{{\small FF3N ``Physics \& Astronomy (4 Years)" at Durham University.}}
\end{center}

\vspace{20mm}
\begin{abstract}
\noindent    A physically-motivated, simple, supernova-driven model of the interstellar medium (ISM) is presented.
    Such a physically-motivated model has proved elusive due to the complex, hierarchical structure of the ISM.
    To combat this, a `zoom-out' approach is suggested, where small-scale, high-resolution ($\approx 1 \pc$) simulations are used to model the self-interaction of supernovae and their interactions with the ISM.
    From these simulations, macroscopic parameters are extracted that are used to promote the well-studied star formation law, $\dot{\sst} \propto \sg (\sg + \sst)$, that results from this model to an equation of state $p \propto \rho_g^{5/4}$.
    The equation of state shows a high level of convergence with the one used in the EAGLE simulations and is tested using a simulation suite generated with a modified version of the Gadget-2 SPH code.
    The model produces marginally (Toomre) stable galaxy disks with a centrally concentrated  mass profile and spiral structure.
    It also predicts that high-$z$ galaxies should be clumpier than their local counterparts due to a dependence of the pressure on gas fraction.
    It is concluded that more work is required to accurately model the ISM as the supernova-driven equation of state neglects significant pressure sources.
\end{abstract}

\end{titlepage}

\newpage

\pagenumbering{arabic}
