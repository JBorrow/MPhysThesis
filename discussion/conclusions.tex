In this work a zoom-out model of the interstellar medium where pressure support is provided by turbulence from supernovae and calibrated using small-scale simulations by \citet{martizzi_supernova_2015} was presented.
One of the well-studied successes of this model is that it fits the data in the $\sst$, $\sg$ plane (a Kennicutt-Schmidt type law) well.
After calibrating the dispersion and testing the equation of state in N-body simulations, it is also clear that this model produces stable disks with a scale height of the correct order ($\approx 100$ pc) that are stable with a theoretically supported Toomre $Q$ value of 0.7 and a noticeable spiral structure.
The results of the N-body simulations were also reasonably resolution-independent, producing similar scale heights and values for the Toomre $Q$ parameter as a function of radius.
The model also manages to produce pressure support that scales inversley with gas fraction, implying that high-redshift galaxies should be clumpier than ones in the local universe.

The supernova-driven model is incomplete, however.
The pressure support provided by this model is not enough to produce realistic galaxies with the majority of the mass in the galaxy being transported into the centre; this is shown by both theoretical work and the N-body simulations, and is aggrevated somewhat by the lack of star formation prescription in the code.

A key observation from this work is that the way that the disks stabalise is by transporting mass into the centre.
Under a situation where an isolated galaxy is unstable due to an unfavourable distribution of mass, the only way to increase the stability is to move mass to the centre (the lowest energy state).
This casts doubt on work that suggests that galaxies are stabalised by gravitational infall of mass; this is more of a consequence of moving to a stable density profile (which depends on the gas properties and underlying dark matter structure) than it is a direct contributor of turbulence and energy to stabalise the galaxy.

It remains to be seen if the addition of extra pressure and energy components (such as a star formation prescription, AGN feedback, etc.) could allow a similar model to produce a realistic galaxy population in a cosmological simulation.
Alone, without these extra components, it is unlikely that the model presented in this work provides enough pressure support to do so.
