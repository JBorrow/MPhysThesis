In this work, a stellar feedback driven model for the gas in a galaxy was studied.
The model has some clear successes, being physically motivated and calibrated from observations and high resolution simulations, but also has failed somewhat on some key tests.

The galaxies produced by the N-body simulations show that there is too little pressure support to produce disks that have a physically accurate scale height and density profile.
To create a stable disk, the galaxies must transport over 80\% of their mass into the centre, which is clearly not reproduced in real-life galaxies.
This, however, could just be a simulation artefact and not an effect from the model as with a star formation prescription a large amount of this mass would be converted to stars due to the high local density in these regions (and hence short gas consumption timescale).
This situation presents a new issue, though, in that if star formation is programmed in to the model then there would be very little gas left in the galaxy (with $\fgas \rightarrow 0$ over one gas consumption timescale in the central regions). <WORK OUT THE GAS CONSUMPTION TIMESCALE IN THE CENTER AND PUT THAT IN HERE SOMEWHERE>.

