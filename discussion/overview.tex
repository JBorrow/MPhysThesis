In this work, a stellar feedback driven model for the gas in a galaxy was studied.
The model has some clear successes, being physically motivated and calibrated from observations and high resolution simulations, but also has failed somewhat on some key tests.

The galaxies produced by the N-body simulations show that there is too little pressure support to produce disks that have a physically accurate scale height and density profile.
To create a stable disk, the galaxies must transport over 80\% of their mass into the centre, which is clearly not reproduced in real-life galaxies.
This, however, could just be a simulation artefact and not an effect from the model as with a star formation prescription a large amount of this mass would be converted to stars due to the high local density in these regions (and hence short gas consumption timescale).
This situation presents a new issue, though, in that if star formation is programmed in to the model then there would be very little gas left in the galaxy (with $\fgas \rightarrow 0$ over one gas consumption timescale in the central regions). <WORK OUT THE GAS CONSUMPTION TIMESCALE IN THE CENTER AND PUT THAT IN HERE SOMEWHERE>.

One success is that the model manages to produce a transient spiral structure from an isotropic initial condition.
This model also shows how mass manages to make its way to the centre of a galaxy continually, with the stable surface density far out in the disk being very low.
This means that for the disk to stabilise it must form clumps and transport them to the centre.
As the clumps are moving towards the centre of the galaxy over a $\approx 2-4$ Gyr timescale, they transform the majority of their mass to stars (WORK OUT THE TYPICAL DENSITY OF A CLUMP AND HENCE STAR FORMATION TIMESCALE).
Because of the lack of hydrodynamic forces on these stars (the supernove candidates that form in the clumps will also blow most of the gas away or vertically out of the disk) they are considerably more Toomre stable. <THIS NEEDS MORE THOUGHT>
However, for this conjecture to be studied in more detail, considerably more physics must be included in the simulation.
In future work, a star formation prescription and stochastic supernovae should be included, with the radiation field from star formation being used to provide the extra pressure to the disk.

