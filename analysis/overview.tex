\begin{figure}[!ht]
    \savebox{\graphicsbox}{\includegraphics[trim={1.95cm 1.9cm 1.9cm 1.9cm}]{sd_simulation.pdf}}% Store image
    \leavevmode\rlap{\usebox{\graphicsbox}}% Set image with complete overlap
    \begin{minipage}[b]{0.63\wd\graphicsbox}% Insert caption
        \caption{Testing}
        \label{fig:toomqsimbigfig}
    \end{minipage}\hspace*{0.33\wd\graphicsbox}% Add space to represent figure width
\end{figure}

The simulations in \S\ref{chap:sims} are analysed using a custom python module, \sv. 
More details about the technical implementation of the analysis module are provided in appendix \ref{app:survis}.
Using \sv, the following aspects of the simulations are studied
\begin{itemize}
\item The \citet{toomre_on_1964} parameter of the galaxy and its evolution,
\item The evolution of the surface density (in particular the gaseous surface density) with time,
\item The evolution of scale height of the disks, as well as the difference between scale height and local \citet{jeans_stability_1902} length.
\end{itemize}
Through these aspects of the simulations it is possible to uncover the stability of the disks with time, their overall properties, and how it is that they become stable relative to the initial conditions.


