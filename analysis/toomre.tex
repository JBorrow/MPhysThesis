\begin{figure}[!ht]
    \savebox{\graphicsbox}{\includegraphics[trim={1.95cm 1.9cm 1.9cm 1.9cm}]{toomre_q_simulation.pdf}}% Store image
    \leavevmode\rlap{\usebox{\graphicsbox}}% Set image with complete overlap
    \begin{minipage}[b]{0.63\wd\graphicsbox}% Insert caption
        \caption{The Toomre $Q$ of of all of the runs from Table \ref{tab:sims} is shown after $\approx 2$ Gyr of evolution, with the colour encoding $Q_{gas}$ being capped at $Q=2$ to ensure clarity. The {\tt custom} model shows clear spiral structure and is marginally stable in accordance with the work of \citet{hopkins_stellar_2012}.}
        \label{fig:toomqsimbigfig}
    \end{minipage}\hspace*{0.33\wd\graphicsbox}% Add space to represent figure width
\end{figure}

\begin{figure}[!ht]
    \includegraphics[trim={1.95cm 0cm 1.9cm 1.33cm}]{Q_evo_simulation.pdf}
    \caption{The evolution of the Toomre $Q$ parameter in the {\tt custom} run is shown. Colour encodes $Q_{gas}$, and is cut off above $2$ for clarity. The disk is initially highly Toomre-unstable, with the outer regions being `overstable' and the inner regions being highly `understable'. This causes a high amount of shear in the disk, leading to clump formation. These clumps, which themselves are highly Toomre unstable, are then transported to the centre of the disk, loosing mass as they do to form spiral-like structures which stabilise the disk. The lack of stability stems from the initial conditions having a radically different mass profile to that of a stable galaxy (see Figure \ref{fig:toomreqthr_dat})}
    \label{fig:toomqsimsmallfig}
\end{figure}
l
The evolution of the Toomre $Q$ parameter in the simulated galaxies is one of the most promising outputs of this model.
Noticably, in the {\tt custom} runs shown in Figure \ref{fig:toomqsimbigfig} produce marginally stable galaxies after around $\approx 1$ Gyr of evolution.
The overall evolution of the Toomre $Q$ parameter over time is shown in Figure \ref{fig:toomqsimsmallfig} and shows that the disk becomes stable over time, albeit in a highly chaoitic way.
The initial conditions are so far from the ideal density profile that the disk begins in a highly unstable state with the disk having a completely non-uniform Toomre $Q$ parameter as well as many regions being far from $Q \approx 1$.
This causes the disk to shear and clump, forming unstable regions that are transported to the center to stabalise the galaxy.
The end state of the galaxy has $Q \approx 0.7$ everywhere; this offset (from $Q\approx 1$) is likely due to the approximations made when calculating the velocity dispersion, $\sigma$, from the calibration simulations.

The work of \citet{hopkins_stellar_2012} shows that real galaxies self-stabalise through star formation to be marginally stable.
It is telling then, that this model that uses star formation to support the disk, leads to galaxies that self-regulate to have $Q\approx1$, or at the very least a uniform Toomre $Q$ parameter.
This uniformity is the key metric of success; it ensures that the ratio of shear to pressure support is constant across the disk.
Approximations in the model, or interactions between the stellar and gaseous surfaces being different than expected, will lead to slight offsets in the stable Toomre $Q$, but it being uniform across the disk ensures that there is no realtive shear between shells and as such the disk remains stable.
One of the main ways that the disks manage to self-regulate in such a way is by transporting mass into the center, and this is covered in more detail in \S \ref{sec:anal_sdevo}.
It also follows from the model that the spiral arms should be the sites of star formation, as their higher densities reduce their gas consumption timescales (Equation \ref{eqn:gc}).

The marginally-stable nature of the disks also, promisingly, produces spiral-like featrures in the simulations.
There are transient spiral structures in all of the {\tt custom} runs, in particular the {\tt custom\_gasonly} run has a very strong spiral arm.
These spiral structures are mainly supported through the transport of mass into the center.
The galaxies form Toomre-unstable clumps very quickly ($<0.05$ Gyr), and as these clumps are being moved into the center of the galaxy they are subjected to shocks and shears that tear them apart to form the spiral structure.
This evolution is shown in more detail in Figure \ref{fig:sd_evo_small} for the {\tt custom} model.

\begin{figure}
    \centering
    \includegraphics{toomre_q_data.pdf}
    \caption{The evolution of the surface density of the disks as a function time, with the theoretical predictions shown as a colour map in the background. Note that the {\tt custom} run is shown on the left and the {\tt default} shown on the right. The {\tt custom} colour map in the background has been changed to use that $\Sigma_Q = \sst + \sg$ rather than the more complex form shown in \S \ref{sec:sigrpred}, this is because the low pressure support in the gaseous disk allows it to couple strongly to the considerably more massive stellar component. This coupling prevents the hydrodynamic forces from having their maximal effect, and leads to $\sigma_*/\sigma_g \approx 1$ for the relevant velocity dispersions of the particles. This leads to a lower predicted surface density for the outer regions as the lower support makes them more susceptible to gravitational instability. The low-density regions have almost no pressure support (and it is likely that they will be in the isothermal regime) meaning that their hydrodynamical effects are negligible. Note how the curves tend to relax towards the $Q\approx1$ line, even in the case of the {\tt default} model where the variations in local $Q$ are significantly smaller than in the {\tt custom} model.}
    \label{fig:toomreqthr_dat}
\end{figure}

