The \citet{toomre1964} parameter, $Q_{gas}$ for the gas in the galaxy, is a measure of the stability of the disk.
For a value of $Q<1$, the disk is unstable, with the shear forces being applied due to the differential rotation in the disk overcoming internal pressure forces of the gas.
For a value of $Q>1$, the disk is overstable, with the pressure forces being more than enough to offset the differential rotation.
Whether real galaxy disks should be stable or not during their evolution is still a topic of debate in the community, however \citet{SOMEBODY} finds that real galaxy disks are \emph{marginally stable}, with a value of $Q \approx 1$.
Also, it is expected that Milky-Way type galaxies are stable over long timescales ($>$ Gyr) and as such here a value of $Q \approx 1$ is chosen as values of $Q >> 1$ lead to high levels of vertical instability in galaxy disks.

The \citet{toomre1964} parameter is given as
\begin{equation}
    Q_{gas} = \frac{c_s \kappa}{\pi G \Sigma_Q}~,
    \label{eqn:Q}
\end{equation}
with $c_s$ the sound speed, $\kappa$ the epicyclic rotation frequency (taken to be $\sqrt{2} v/R$ in accordance with \citet{WHOEVER}) and $\Sigma_Q$ a combination of the gaseous and stellar surface densities of the disk.
It is important to note that the stellar and gaseous components of the disk contribute differently to the disks stability \citep{rafikov2001}, and so here
\begin{equation}
    \Sigma_Q = \sg  + \left(\frac{2}{1 + f_\sigma^2}\right)\sst~,
    \label{eqn:sgq}
\end{equation}
with $f_\sigma = \sigma_*/\sigma_g \approx 2$ for real galaxies \citep{korchagin2003}
    
