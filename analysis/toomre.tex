\begin{figure}[!ht]
    \savebox{\graphicsbox}{\includegraphics[trim={1.95cm 1.9cm 1.9cm 1.9cm}]{toomre_q_simulation.pdf}}% Store image
    \leavevmode\rlap{\usebox{\graphicsbox}}% Set image with complete overlap
    \begin{minipage}[b]{0.63\wd\graphicsbox}% Insert caption
        \caption{The Toomre $Q$ of of all of the runs from Table \ref{tab:sims} is shown after $\approx 2$ Gyr of evolution, with the ccolour encoding $Q_{gas}$ being capped at $Q=2$ to ensure clarity. The {\tt custom} model shows clear spiral structure and is marginally stable in accordance with the work of \citet{hopkins_stellar_2012}.}
        \label{fig:toomqsimbigfig}
    \end{minipage}\hspace*{0.33\wd\graphicsbox}% Add space to represent figure width
\end{figure}

The evolution of the Toomre $Q$ parameter in the simulated galaxies is one of the most promising outputs of this model.
Noticably, in the {\tt custom} runs shown in Figure \ref{fig:toomqsimbigfig} produce marginally stable galaxies after around $\approx 1$ Gyr of evolution.
The overall evolution of the Toomre $Q$ parameter over time is shown in Figure \ref{fig:toomqsimsmallfig} and shows that... <MORE HERE>.

The work of \citet{hopkins_stellar_2012} shows that real galaxies self-stabalise through star formation to be marginally stable.
It is telling then, that this model that uses star formation to support the disk, leads to galaxies that self-regulate to have $Q\approx1$.
One of the main ways that the disks manage to self-regulate in such a way is by transporting mass into the center, and this is covered in more detail in \S \ref{sec:sd_evol}.
It also follows from the model that the spiral arms should be the sites of star formation, as their higher densities reduce their gas consumption timescales (Equation \ref{eqn:marttimescale}).

The marginally-stable nature of the disks also, promisingly, produces spiral-like featrures in the simulations.
There are transient spiral structures in all of the {\tt custom} runs, in particular the {\tt custom\_gasonly} run has a very strong spiral arm.
These spiral structures are mainly supported through the transport of mass into the center.
The galaxies form Toomre-unstable clumps on <SOME TIMESCALE> timescales, and as these clumps are being moved into the center of the galaxy they are subjected to shocks and shears that tear them apart to form the spiral structure.
This evolution is shown in more detail in Figure \ref{fig:sd_evo_small} for the {\tt custom} model.

