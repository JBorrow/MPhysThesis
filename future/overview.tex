It is apparent that this field, whilst it has remained relatively neglected by the simulation community, holds a huge amount of merit.
The equation of state that the gas behaves according to causes huge variations in the stability and evolution of galaxies, as well as the physical observables that are reproduced.
Simulators have focused on adding in extra energy sources into simulations, such as AGN feedback, stochastic supernovae, and even radiative transfer to model radiation outputs from stars.
This work has shown that a more computationally efficient, in particular for testing purposes, way of including these in simulations is to build the into the equation of state.
Other authors have recently shown that baryonic physics is hugely important for the evolution and dynamics of galaxies, \citep{schaye_eagle_2015, vogelsberger_introducing_2014}, however this has mainly been included through the use of extra energy sources such as those outlined above.

Moving forward, this author suggests the following avenues for future research:
