In this work, a fixed NFW profile for the dark matter halo was utilised, rather than a `live halo' made up of dark matter particles in the simulation.
This is, of course, non-ideal, however it does avoid some of the complications that arise from having high-mass dark matter particles and provides a quicker code.

In the future, live dark matter halos should be used to test the equation of state presented here.
An idealised halo should not be used; one with large amounts of substructure extracted from a cosmological simulation is ideal.
In the above, it has been shown that the way that the galaxy stabilises is by moving mass into the centre.
Whether this is a physical method to stabilise the galaxy still remains to be seen, however it is still interesting to study in this context.

To transport mass into the centre of the galaxy, first large clumps must form.
These clumps have a relatively strong gravitational attraction and have a column density of order $100 \msun \pc^{-2}$.
The reason that the interaction between the substructure of the live halo and the baryonic matter is interesting is because of the necessity of the clumps above.
These clumps could drag the substructure of the halo into the centre, hence `removing' it and allowing it to merge with the central large halo, which would erase some of the substructure present in the halo.
