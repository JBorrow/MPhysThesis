A fixed NFW profile for the dark matter halo was utilised for the simulations here, rather than a `live halo' made up of dark matter particles in the simulation.
This is clearly an oversimplification of the problem; real halos have substructure - however it does avoid some of the complications that arise from having high-mass dark matter particles and provides a quicker code.

Ideally, live dark matter halos extracted from a cosmological simulation should be used in simulations used to test equations of state.
In the above, it has been shown that the way that the galaxy stabilises is by moving mass into the centre.
To transport mass into the centre of the galaxy, first large clumps must form.
These clumps have a strong gravitational field with a gas column density of order $100 \msun \pc^{-2}$.
The clumps could drag the substructure of the halo into the centre, hence `removing' it and allowing it to merge with the central large halo, which would erase some of the substructure present in the halo.
Whilst this is certainly speculation, it is an interesting possibility and is one avenue for future research.
