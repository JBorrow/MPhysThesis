A star-formation prescription, such as the one suggested in \citet{tasker_simulating_2006}, is required to produce physical results in the study of this model.
It would also allow for the evolution of a gas cloud through to a present-day galaxy as currently the stellar population and gas fraction is fixed; this would give a considerably more useful picture of galactic evolution.
It is clear that, in particular, the high gas fraction ($\fgas$) galaxies studied in the {\tt custom\_gasonly} and {\tt default\_gasonly} runs would have benefited significantly from the ability to form actual stars and reduce their gas fraction (and hence, in the {\tt custom\_gasonly} model, produce a higher level of pressure support).

This would also require that the gas fraction is measured locally, rather than globally, and this would require extra modifications to the SPH code.
