\begin{figure}[!ht]
    \savebox{\graphicsbox}{\includegraphics[trim={1.95cm 1.5cm 1.9cm 1.9cm}]{sd_simulation.pdf}}% Store image
    \leavevmode\rlap{\usebox{\graphicsbox}}% Set image with complete overlap
    \begin{minipage}[b]{0.63\wd\graphicsbox}% Insert caption
        \caption{The surface density (in gas only) of each of the runs from Table \ref{tab:sims} is shown, with colour encoding surface density. Note that the colour map is cut off below $10^{-1} \msun \pc^{-2}$ and above $10^3 \msun \pc^{-2}$ for clarity. Spiral structure has formed in the {\tt custom} runs with density waves permeating the whole galaxy, whereas the {\tt default} runs have remained uniform.}
        \label{fig:sdbig}
    \end{minipage}\hspace*{0.33\wd\graphicsbox}% Add space to represent figure width
\end{figure}

All simulations were ran with the same fixed NFW profile with $c = 20.1$, $R = 16.1$ kpc, and $M_{H} = 1.53 \times 10^{12}$ $\msun$, where $c$ is the concentration parameter such that
$$
    M_{H} = 4\pi \rho_{DM} R^3 \left[ \ln (1 + c) - \frac{c}{1+c} \right]~,
$$
is the mass of the halo with $R$ the scale radius.
The density of the halo as a function of radius is the NFW profile,
\begin{equation}
    \rho(r) = \frac{\rho_{DM}}{\frac{r}{R} \left( 1 + \frac{r}{R} \right)^2}~.
    \label{eqn:nfw}
\end{equation}

The runs also have a fixed total mass in the disk, of $8\times10^{10} \msun$ to reproduce approximately the mass of the Milky Way \citep{nakanishi_three-dimensional_2016}.
The disks are initialised with an exponential profile, with scale radius $3$ kpc for both stellar and gaseous material.
The scale height of the gas is initialised to be 2 kpc (with a sech$^2$ profile), with the stellar component initially having a scale height of 0.3 kpc.
The simulations are shown after $\approx 2$ Gyr of evolution in Figure \ref{fig:sdbig}. 
