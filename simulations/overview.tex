The equation of state (Equation \ref{eqn:eos}) found in this work is implemented in the Gadget-2 SPH code \citep{springel_cosmological_2005} such that N-body simulations of galaxies can be used to test its applicability to a cosmological-scale simulation.
Details of the implementation, as well as the custom initial conditions generator, are available in Appendices \ref{app:isg} and \ref{app:ics} respectively.

It is clear that the most robust method to test any change in gas properties is to run a full N-body simulation of a cosmological volume, in a similar fashion to \citet{vogelsberger_introducing_2014} and \citet{schaye_eagle_2015}.
These simulations, however, are incredibly computationally expensive and whilst a single galaxy simulation will be unable to predict the resultant galaxy population from a cosmological simulation, it will be able to predict the resultant density profile and stability of a galaxy disk.
If it is not possible to form a stable galaxy disk in an isolated case, then it is unlikely that a full cosmological simulation is worthwhile.

Using disk galaxies also make it possible to test the theory of \citet{schaye_model-independent_2001} that $\sg \approx \rho_g L_J$, and hence provide a test for the methodology used to generate the equation of state.

The simulations utilise a fixed NFW profile \citep{navarro_structure_1996, coe_dark_2010} to model the dark matter.
This was chosen to prevent spurious fluctuations caused by high-mass dark matter particles, as well as provide a more computationally efficient code.
