\section{Proofs}\label{app:proofs}

\subsection{$p_T = F \dot{\Sigma}_*\left(\frac{P_{fin}}{m_*}\right)$}\label{proof-of-p_t-f-dotsigma_leftfracp_finm_right}

This equation is rather complicated at a first glance; why should this
be the case? One can consider dimensional grounds:
\[
    [p_T] = \frac{[M][L]}{[T]^2[L]^2}~,
\]
\[
    [\dot{\Sigma}_*] = \frac{[M]}{[L]^2[T]}~,
\]
\[
    \left[\frac{P_{fin}}{m_*}\right] = \frac{[L]}{[T]}~,
\]
from which it is obvious that the `units match' and there must just be
some numerical factor \(F\) that `makes it work'. However, this is not a
satisfactory `proof', and some conceptual insight would be helpful so
that we can consider where this equation is appropriate.

Under normal conditions, at least in disk galaxies, one would assume
that the supernova rate (here \(\dot{\Sigma}_s\)) is simply proportional
to the star formation rate (i.e.~stars that die are replaced by an equal
number of stars over time)\footnote{note that this corresponds to a rate
of change of surface density of each respective variable}.
\[
    \dot{\Sigma}_s = K_1 \dot{\Sigma}_*~.
\]
To deal with the fact that this is a surface density, instead of a
surface number density, we will set
\[
    m_s = K_2 m_*~,
\]
and hence
\[
    \dot{\Sigma}_{s, ~num. ~density} = \frac{K_1}{K_2} \dot{\Sigma}_*~.
\]
If each supernova injects \(P_{fin}\) of momentum, then the rate of
momentum injection (i.e.~a force) by \(n\) supernovae is (up to an
efficiency factor \(K_3\))
\[
    F = K_3\dot{n}P_{fin}~,
\]
Now, using the above surface number density, we can retrieve an effective
pressure:
\[
    p_T = \frac{K_1 K_3}{K_2} \dot{\Sigma}_* \left(\frac{P_{fin}}{m_*}\right)~,
\]
which recovers Marizzi's result. The issue here is that \citet{martizzi_supernova_2016}
argues that the factor,
\[
    F = \frac{K_1 K_3}{K_2}~,
\]
is of order unity. Is this reasonable? Let us examine:
\[
    F = \frac{(\mathrm{SFR ~to ~SNR}) (\mathrm{efficiency ~of ~p ~to ~F ~conversion})}{(\mathrm{m_s/m_*})}~.
\]
and as such it is reasonable that $F$ should be of \emph{order} unity.


\subsection{$p_T \approx \pi G \sg \left(\sg + \sst\right)$}

From hydrostatic equilibrium\footnote{This could be expanded a little}, the weight of the disk
$$
	W = \int^\infty_0 \rho_g \left(\der{\Phi_g + \Phi_b}{z}\right) \mathrm{d}z = W_1 + W_2~,
$$
The first (self gravity) term can be written using the Poisson equation
$$
	W_1 = \frac{1}{8\pi G}\int^\infty_0 \der{\left(\der{\Phi_g}{z}\right)^2}{z} \mathrm{d} z = \frac{\pi G \sg^2}{2}~,
$$
with the second half needing to be solved in a less elegant manner,
$$
	W_2 = \int^\infty_0 \rho_g \der{\Phi_b}{z}\mathrm{d}z = \frac{4\pi G \rho_b}{3} \int^\infty_0 \rho z \mathrm{d}z = \frac{2 \pi \kappa G \rho_b \sg^2}{3 \rho_0} \approx \pi G \sg \sst
$$
leading to
$$
	p_T \approx \pi G \sg\left(\sg + \sst\right)~.
$$


\subsection{Equation of state from supernova driven model}

Beginning with the calibrated velocity dispersion from \citet{martizzi_supernova_2015}
$$
	\sigma_{mod} = 1.8 \left(\frac{f}{F}\right)^{{3/5}} G^{{2/5}} P_{fin}^{{1/5}} \fgas^{{-2/5}} \sg^{{1/5}}~.
$$
From \citet{schaye_star_2004}, it is possible to write
$$
	\sg = \rho_g L_J = \sigma_{mod} \sqrt{\frac{9 \rho_g}{8G}}~,
$$
and substituting this
$$
\sigma_{mod} = 1.8^{4/5} \left(\frac{f}{F}\right)^{{3/4}} G^{{1/2}} P_{fin}^{{1/4}} \fgas^{{-1/4}} \rho_g^{{1/4}} \left(\frac{9}{8G}\right)^{1/4}~.
$$
Now, taking $p = \rho_g \sigma_{mod}^2$, we recover
$$
p = 4.5\left(\frac{f}{F}\right)^{{3/2}} G^{{3/4}} P_{fin}^{{1/2}} \fgas^{-1} \rho_\mathrm{g}^{{5/4}}~.
$$
