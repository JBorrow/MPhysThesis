For this work, a modified version of the Gadget-2 simulation code \citep{springel_cosmological_2005} was
used\footnote{This code is available at
  \url{https://github.com/JBorrow/InterStellarGadget}}. This code has a fixed
NFW profile potential, reducing issues with large mass dark matter
particles or a large amount of slowdown due to unnecessary gravity
calculations. This was implemented by adding an acceleration term in the
gravity tree portion of Gadget.

Simulations also require initial conditions. For this work, a custom
generator\footnote{Available at \url{https://github.com/JBorrow/GoGoGadget}}
was used, which is described above in Appendix \ref{app:ics}. This generates exponential disks for the stellar and gaseous
disks with the density profile

\begin{equation}
\rho \propto \exp\left(-\frac{r}{R}\right)\rm{sech}^2\left(\frac{z}{Z}\right)~,
\end{equation}
which are summarised in \citet{ferriere_interstellar_2001}.

The simulations in this work have configuration files that are
available on GitHub\footnote{\url{https://github.com/JBorrow/MPhysProjectConfigs}} and were simulated on the
Hamilton HPC service and COSMA HPC service at Durham University. Datasets are not made
available due to their prohibitive size, however some moves are
available in the aforementioned repository.

\subsection{Gadget Time}
\label{app:gadgettime}

It is important to be able to convert between snapshot number, internal code time, and `real' time.
An easy way to calculate this is to take the units of velocity ($\kms$) and distance (kpc) used in the code to calculate timesteps,
$$
    v ~ [\kms] ~ \cdot \Delta t ~ [\mathrm{sim}] = x ~ [\mathrm{kpc}]~.
$$
This leads to a relation
$$
    \mathrm{sim} = \frac{\pc}{\mathrm{m}} \mathrm{s}~,
$$
meaning that one simulation unit of time $[\mathrm{sim}] \approx 1 \mathrm{Gyr}$, and each snapshot encapsulates 0.02 Gyr of `real' time.
