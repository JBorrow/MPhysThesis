\subsection{Turbulent}

The following derivation is for the turbulent Jeans' length for a cloud.
Begin with the virial theorem,
$$
    KE = - \frac{1}{2} GPE~,
$$
with $KE$ the kinetic energy and $GPE$ the gravitational potential energy,
$$
\frac{1}{2} m (3 \sigma^2) \approx \frac{1}{2} \frac{G M m}{(1/2)R}~,
$$
with $m$ the mass of a test particle, $\sigma$ the 1D velocity dispersion of the particles (it is assumed that this is isotropic for this simple approximation), $M$ the mass of the cloud and $R$ the radius.
When these are equal, the radius is the Jeans' length,
$$
L_{J, \sigma} = \frac{2 G M}{3 \sigma^2}~,
$$
and trading out the $M = (4/3) \rho L_{J, \sigma}^3$ with $\rho$ the local 3D density,
$$
L_{J, \sigma} = \sigma \sqrt{\frac{9}{8 G \rho}}~.
$$

\subsection{Thermal}

The following derivation is for the turbulent Jeans' length for a cloud.
Begin with the virial theorem,
$$
    KE = - \frac{1}{2} GPE~,
$$
with the KE coming from the thermal pressure and GPE coming from the gravitational potential of a spherical cloud,
$$
    \frac{3 M kT}{2 m} = \frac{3 G M^2}{5R}~,
$$
with $m$ the molecular mass, and $kT$ the thermal energy.
Plugging in the above and taking $R = L_{J, T}$ here,
$$
    L_{J, T} = \sqrt{\frac{15 kT}{4\pi\rho G m}}~.
$$
