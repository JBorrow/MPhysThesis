\section{Generating Initial
Conditions}\label{app:gen}

To simulate galaxies, initial conditions must be generated such that they
can be time-evolved with GADGET. These initial conditions need not be
perfect, as the simulation can be ran for a few calibration timescales
so that parameters stabilise, but a rough approximation
of a disks gas and dark matter with their associated velocities is required to
begin a simulation.

\subsubsection{Particle Positions}

The probability of finding a particle at some position \(\vec{x}\) is
given by
\[
    p(\vec{x})\mathrm{d}\vec{x} = \rho(\vec{x})\mathrm{d}V~,
\]
where \(\mathrm{d}V\) is a volume element. There are several important
density profiles given in Table \ref{tab:profiles}.

\begin{table}
\centering
\begin{tabular}{c|c|c|c}
Name   &  Type of Matter  &    Density               & System of Co-ordinates \\ \hline
NFW Profile & Dark matter & $\rho(r) \propto \frac{R}{r\left(1 + \frac{r}{R}\right)^2}$ & Spherical polar \\
Radial Gas  & Gas & $\rho(r) \propto e^{-r/R}$ & Cylindrical polar \\
Vertical Gas & Gas & $\rho(z) \propto \mathrm{sech}^2\left(\frac{z}{z_0}\right)$ & Cylindrical polar \\
\end{tabular}
\caption{Various density profiles for the disks and the NFW profile \citep{ferriere_interstellar_2001, coe_dark_2010}.}
\label{tab:profiles}
\end{table}

It is assumed that all other profiles are uniform (for example,
\(\theta,\) \(\phi\) for the NFW profile). To actually generate the
particle distributions, though, the jacobians such that
\[
    p(r)\mathrm{d}r = \rho(r) f(r) \mathrm{d}r~,
\]
are required, where \(f(r)\) is some arbitrary function of \(r\) that is given by the
jacobian. Here these are
\[
    \mathrm{d}V_{r, \theta, \phi} = r^2\sin\phi\mathrm{d}r\mathrm{d}\theta\mathrm{d}\phi ~ ~ ~ \rightarrow ~ ~ ~ \mathrm{d}V_{r} = 4\pi r^2~,
\] \[
    \mathrm{d}V_{r, \theta, z} = r\mathrm{d}r\mathrm{d}\theta\mathrm{d}z ~ ~ ~ \rightarrow ~ ~ ~ \mathrm{d}V_{r} = 2\pi r \mathrm{d}r ~ ~ ~ \mathrm{d}V_z = 2\pi z \mathrm{d}z~.
\]

It is worth noting that the stellar profiles are distributed similarly
to the gas profiles, with different scale radii and heights. In this work, the NFW profile (and hence dark matter profiles) were not simulated and were hard-coded into the GADGET software (see Appendix \ref{app:sims}).

\subsection{Invertible Distributions: Analytical
Solution}\label{easy-distributions---analytical-solution}

For the \(\rho(z)\) gas profile, it is relatively simple to generate
numbers that follow this distribution. This is performed by finding the
cumulative distribution function,
\[
    P(z) = \int_{-\infty}^z p(z') \mathrm{d}z'~,
\]
which by definition is bounded by {[}0, 1{]}. Hence it is possible to invert the
equation for \(P(z)\), i.e.~find \(z(P)\), and generate random numbers
uniformly (\(U(0, 1)\)), which is trivial computationally, and calculate
\(z(U(0, 1))\) which will generate random numbers with the distribution
\(p(z)\)\footnote{Clearly, these are pseudorandom numbers, but this is
  not important for the discussion here.}.

This requires that it is possible to:
\begin{itemize}
\item
  Calculate \(P(z)\) analytically
\item
  Invert \(P(z)\),
\end{itemize}
which clearly is not a given. For a more complicated distribution, the
rejection method is required.

\subsection{Non-Invertible Distributions: The Rejection
Method}\label{hard-distributions---the-rejection-method}

The rejection method relies on being able to find some distribution
\(f(x)\) that is always above the model distribution \(p(x)\) which in theory
should always be possible as \(p(x)\) must be normalisable, however
in practice this is somewhat complicated. The algorithm to generate
random variates with the distribution of \(p(x)\) is then as follows:

\begin{enumerate}
\item
  Generate random variate \(x\) from \(f(x)\)
\item
  Generate \(f(x) \cdot U(0, 1) = y\)
\item
  If \(y \leq p(x)\) keep \(x\) else reject \(x\).
\end{enumerate}

This has the efficiency \(p(x)/f(x)\) where these values represent the
total areas under the curves. It is also useful if \(f(x)\) has an
invertible \(F(x)\) such that it is easy to generate numbers within this
distribution, else the rejection method must be used to find variates
\emph{those} which becomes very computationally expensive \citep{press_numerical_2007}.

